\documentclass[10pt,aspectratio=169]{beamer}

% --- Packages ---
\usepackage[utf8]{inputenc}
\usepackage[T1]{fontenc}
\usepackage[french]{babel}
\usepackage{amsmath, amssymb, amsfonts}
\usepackage{booktabs}
\usepackage{tikz}
\usepackage{pgfplots}
\pgfplotsset{compat=1.18}
\usepackage{graphicx}
\usepackage{array}
\usepackage{colortbl}
\usepackage{lmodern}
\usepackage{tcolorbox}
\tcbuselibrary{skins}
\usetikzlibrary{matrix, positioning, calc, shadows.blur, shapes.geometric, arrows.meta, fit, backgrounds}

% --- Color Palette (Modern & Professional) ---
\definecolor{primary}{RGB}{0, 122, 204}
\definecolor{secondary}{RGB}{240, 242, 245}
\definecolor{accent}{RGB}{255, 87, 34}
\definecolor{darktext}{RGB}{33, 33, 33}
\definecolor{lighttext}{RGB}{117, 117, 117}
\definecolor{success}{RGB}{46, 204, 113}
\definecolor{danger}{RGB}{231, 76, 60}
\definecolor{warning}{RGB}{241, 196, 15}

% --- Theme Settings ---
\usefonttheme{professionalfonts}
\setbeamerfont{title}{size=\huge,series=\bfseries}
\setbeamerfont{subtitle}{size=\large,series=\normalfont}
\setbeamerfont{frametitle}{size=\Large,series=\bfseries}
\setbeamerfont{block title}{series=\bfseries}

% Colors
\setbeamercolor{background canvas}{bg=white}
\setbeamercolor{normal text}{fg=darktext}
\setbeamercolor{frametitle}{fg=primary}
\setbeamercolor{title}{fg=primary}
\setbeamercolor{subtitle}{fg=lighttext}
\setbeamercolor{structure}{fg=primary}
\setbeamercolor{item}{fg=accent}

% Navigation & Footline
\setbeamertemplate{navigation symbols}{}
\setbeamertemplate{footline}{
    \begin{tikzpicture}[remember picture,overlay]
        \node[anchor=south east, xshift=-10pt, yshift=10pt, text=lighttext, font=\scriptsize] at (current page.south east) {
            \insertframenumber/\inserttotalframenumber
        };
        \fill[primary] (current page.south west) rectangle ($(current page.south west)+(10pt,10pt)$);
    \end{tikzpicture}
}

% --- Custom Blocks ---
\newtcolorbox{modernblock}[1]{
    colback=primary!5!white,
    colframe=primary,
    coltitle=white,
    boxrule=0.5pt,
    arc=4pt,
    title={#1},
    fonttitle=\bfseries
}

\newtcolorbox{alertblockbox}[1]{
    colback=accent!5!white,
    colframe=accent,
    coltitle=white,
    boxrule=0.5pt,
    arc=4pt,
    title={#1},
    fonttitle=\bfseries
}

\newtcolorbox{exampleblockbox}[1]{
    colback=success!5!white,
    colframe=success,
    coltitle=white,
    boxrule=0.5pt,
    arc=4pt,
    title={#1},
    fonttitle=\bfseries
}

% --- Custom Frametitle ---
\setbeamertemplate{frametitle}{
    \vspace{0.5cm}
    \begin{tikzpicture}[remember picture, overlay]
        \node[anchor=west, text=primary, font=\Large\bfseries] at (0,0) {\insertframetitle};
        \draw[thick, accent] (0,-0.5) -- (1,-0.5);
        \draw[thin, lightgray] (1,-0.5) -- (\textwidth,-0.5);
    \end{tikzpicture}
    \vspace{0.5cm}
}

% --- Metadata ---
\title{Chiffrement Chaotique d'Images}
\subtitle{Implémentation et Analyse de Sécurité}
\author{Hicham Moussaid \& Ahmed [NOM] \& Mohamed [NOM]}
\date{Année Universitaire 2024–2025}
\institute{Master SIE — ENS Meknès — Encadré par: Pr. Samir OMARI}

\begin{document}

% --- Title Page ---
{
\setbeamertemplate{footline}{}
\begin{frame}[plain]
\begin{tikzpicture}[remember picture, overlay]
    \fill[white] (current page.south west) rectangle (current page.north east);
    \fill[primary!10] (current page.north east) circle (5cm);
    \fill[accent!10] (current page.south west) circle (4cm);
    
    \node[anchor=center, align=center] at (current page.center) {
        \vspace{-1cm}
        {\Huge\bfseries\textcolor{primary}{Chiffrement Chaotique}}\\[0.2cm]
        {\huge\bfseries\textcolor{darktext}{d'Images}}\\[0.5cm]
        {\Large\textcolor{accent}{Implémentation et Analyse de Sécurité}}\\[1.5cm]
    };
    
    \node[anchor=south west, xshift=1cm, yshift=1cm, align=left] at (current page.south west) {
        \textbf{\textcolor{primary}{Présenté par :}}\\[0.1cm]
        \small Hicham Moussaid\\
        \small Ahmed [NOM]\\
        \small Mohamed [NOM]
    };
    
    \node[anchor=south east, xshift=-1cm, yshift=1cm, align=right] at (current page.south east) {
        \textbf{\textcolor{primary}{Encadré par :}}\\[0.1cm]
        \small Pr. Samir OMARI
    };
    
    \node[anchor=south, yshift=0.3cm, text=lighttext, font=\scriptsize] at (current page.south) {
        Master SIE — ENS Meknès | Année Universitaire 2024–2025
    };
\end{tikzpicture}
\end{frame}
}

% --- Plan ---
\begin{frame}{Plan de la Présentation}
    \centering
    \begin{tikzpicture}
        \node[draw=primary, fill=white, rounded corners, minimum width=2.5cm, minimum height=1cm, thick] (intro) at (0,0) {\textbf{1. Introduction}};
        \node[draw=primary, fill=white, rounded corners, minimum width=2.5cm, minimum height=1cm, thick] (theory) at (3.5,0) {\textbf{2. Théorie}};
        \node[draw=primary, fill=white, rounded corners, minimum width=2.5cm, minimum height=1cm, thick] (example) at (7,0) {\textbf{3. Exemple}};
        \node[draw=primary, fill=white, rounded corners, minimum width=2.5cm, minimum height=1cm, thick] (impl) at (10.5,0) {\textbf{4. Implémentation}};
        
        \draw[->, thick, accent] (intro) -- (theory);
        \draw[->, thick, accent] (theory) -- (example);
        \draw[->, thick, accent] (example) -- (impl);
        
        \node[below=0.2cm of intro, align=center, font=\footnotesize, text=lighttext] {Contexte\\Problématique};
        \node[below=0.2cm of theory, align=center, font=\footnotesize, text=lighttext] {Cartes\\Algorithme};
        \node[below=0.2cm of example, align=center, font=\footnotesize, text=lighttext] {Calculs\\Détaillés};
        \node[below=0.2cm of impl, align=center, font=\footnotesize, text=lighttext] {Résultats\\Analyse};
    \end{tikzpicture}
    
    \vspace{0.5cm}
    \begin{tikzpicture}
        \node[draw=primary, fill=white, rounded corners, minimum width=2.5cm, minimum height=1cm, thick] (security) at (0,0) {\textbf{5. Sécurité}};
        \node[draw=primary, fill=white, rounded corners, minimum width=2.5cm, minimum height=1cm, thick] (results) at (3.5,0) {\textbf{6. Résultats}};
        \node[draw=primary, fill=white, rounded corners, minimum width=2.5cm, minimum height=1cm, thick] (conclusion) at (7,0) {\textbf{7. Conclusion}};
        
        \draw[->, thick, accent] (security) -- (results);
        \draw[->, thick, accent] (results) -- (conclusion);
        
        \node[below=0.2cm of security, align=center, font=\footnotesize, text=lighttext] {NPCR\\Entropie};
        \node[below=0.2cm of results, align=center, font=\footnotesize, text=lighttext] {Graphiques\\Comparaison};
        \node[below=0.2cm of conclusion, align=center, font=\footnotesize, text=lighttext] {Synthèse\\Perspectives};
    \end{tikzpicture}
\end{frame}

% --- Section Introduction ---
\section{Introduction}

\begin{frame}{Contexte : Sécurité des Images Numériques}
    \begin{columns}[T]
        \begin{column}{0.58\textwidth}
            \begin{itemize}
                \item[\textcolor{primary}{$\bullet$}] \textbf{Explosion des données visuelles}
                \begin{itemize}
                    \footnotesize
                    \item Réseaux sociaux, télémédecine, IoT
                    \item Nécessité de protection
                \end{itemize}
                
                \vspace{0.3cm}
                \item[\textcolor{primary}{$\bullet$}] \textbf{Faiblesses des systèmes standards (DES, AES, RSA)}
                \begin{itemize}
                    \footnotesize
                    \item Inadéquats pour les données d'images (blocs indépendants)
                    \item \textbf{Exposition aux attaques statistiques}
                    \item Corrélation spatiale et redondance non traitées
                \end{itemize}
                
                \vspace{0.3cm}
                \item[\textcolor{primary}{$\bullet$}] \textbf{Solution : Technologie basée sur le chaos}
                \begin{itemize}
                    \footnotesize
                    \item Systèmes non linéaires, déterministes et apériodiques
                    \item Très grande sensibilité aux conditions initiales
                    \item Signal imprévisible à long terme
                \end{itemize}
            \end{itemize}
        \end{column}
        
        \begin{column}{0.38\textwidth}
            \centering
            \begin{tikzpicture}[scale=0.65]
                \fill[primary!10] (0,0) rectangle (4,4);
                \draw[step=1, thick, white] (0,0) grid (4,4);
                
                \foreach \x/\y/\val in {0.5/3.5/120, 2.5/3.5/80, 1.5/2.5/200, 3.5/2.5/50} {
                    \node[font=\small\bfseries, text=darktext] at (\x,\y) {\val};
                }
                
                \node[below=0.3cm, font=\small\bfseries, text=primary] at (2,0) {Image Originale};
            \end{tikzpicture}
            
            \vspace{0.3cm}
            \begin{tikzpicture}[scale=0.65]
                \fill[accent!10] (0,0) rectangle (4,4);
                \draw[step=1, thick, white] (0,0) grid (4,4);
                
                \foreach \x/\y/\val in {0.5/3.5/234, 2.5/3.5/12, 1.5/2.5/189, 3.5/2.5/67} {
                    \node[font=\small\bfseries, text=darktext] at (\x,\y) {\val};
                }
                
                \node[below=0.3cm, font=\small\bfseries, text=accent] at (2,0) {Image Chiffrée};
            \end{tikzpicture}
        \end{column}
    \end{columns}
\end{frame}

\begin{frame}{Principes Fondamentaux de Sécurité (Shannon)}
    \begin{columns}[T]
        \begin{column}{0.48\textwidth}
            \begin{modernblock}{Principe de Kirchhoff}
                \footnotesize
                La sécurité d'un système doit reposer uniquement sur le secret de la \textbf{clé} et non sur l'algorithme.
            \end{modernblock}
            
            \vspace{0.3cm}
            \textbf{\textcolor{primary}{Propriétés de Shannon}}
            \begin{itemize}
                \item[\textcolor{accent}{$\bullet$}] \textbf{Confusion} : Création d'une relation complexe entre le clair et la clé.
                \item[\textcolor{accent}{$\bullet$}] \textbf{Diffusion} : Un petit changement dans le clair doit affecter tout le chiffré.
            \end{itemize}
        \end{column}
        
        \begin{column}{0.48\textwidth}
            \centering
            \begin{tikzpicture}[scale=0.8]
                \node[draw, primary, thick, rounded corners] (k) at (0,0) {Cryptographie Moderne};
                \node[draw, accent, thick, rounded corners, below left=0.5cm of k] (c) {Confusion};
                \node[draw, accent, thick, rounded corners, below right=0.5cm of k] (d) {Diffusion};
                \draw[->, thick, primary] (k) -- (c);
                \draw[->, thick, primary] (k) -- (d);
                
                \node[below=0.3cm of c, font=\tiny, align=center] {Permutation / \\ Substitution};
                \node[below=0.3cm of d, font=\tiny, align=center] {Technique de \\ Chaînage (CBC)};
            \end{tikzpicture}
        \end{column}
    \end{columns}
\end{frame}

\begin{frame}{Objectifs du Projet}
    \begin{exampleblockbox}{Objectifs Principaux}
        \begin{enumerate}
            \item \textbf{Implémenter} un système de chiffrement chaotique complet
            \item \textbf{Utiliser} 3 cartes chaotiques (Logistique, Tente, PWLCM)
            \item \textbf{Analyser} la sécurité (NPCR, UACI, Entropie, Corrélation)
            \item \textbf{Développer} une application Web Full-Stack moderne
        \end{enumerate}
    \end{exampleblockbox}
    
    \vspace{0.8cm}
    \centering
    \textbf{Architecture Globale :}
    
    \vspace{0.4cm}
    \begin{tikzpicture}
        \node[fill=primary!10, draw=primary, rounded corners=3pt, minimum width=2.5cm, minimum height=1cm, align=center, font=\bfseries] (chaos) at (0,0) {Cartes\\Chaotiques};
        \node[fill=accent!10, draw=accent, rounded corners=3pt, minimum width=2.5cm, minimum height=1cm, align=center, font=\bfseries] (sbox) at (3.5,0) {S-Box\\Dynamique};
        \node[fill=success!10, draw=success, rounded corners=3pt, minimum width=2.5cm, minimum height=1cm, align=center, font=\bfseries] (cbc) at (7,0) {Diffusion\\CBC};
        \node[fill=secondary!50, draw=darktext, rounded corners=3pt, minimum width=2.5cm, minimum height=1cm, align=center, font=\bfseries] (result) at (10.5,0) {Image\\Chiffrée};
        
        \draw[->, thick, primary] (chaos) -- (sbox);
        \draw[->, thick, accent] (sbox) -- (cbc);
        \draw[->, thick, success] (cbc) -- (result);
    \end{tikzpicture}
\end{frame}

% --- Section Fondements ---
\section{Fondements Théoriques}

\begin{frame}{Cartes Chaotiques - Unidimensionnelles}
    \scriptsize
    \renewcommand{\arraystretch}{1.3}
    \begin{table}
        \centering
        \begin{tabular}{|l|c|l|}
            \hline
            \rowcolor{primary!20}
            \textbf{Carte} & \textbf{Expression} & \textbf{Paramètres} \\
            \hline
            \textbf{Logistique} & $x_{n+1} = \mu x_n(1 - x_n)$ & $\mu \in [3.57, 4], x_0 \in ]0, 1[$ \\
            \hline
            \textbf{Tente} & $x_{n+1} = \begin{cases} r x_n & x_n < 0.5 \\ r(1 - x_n) & x_n \geq 0.5 \end{cases}$ & $r \in ]0, 2], x_0 \in ]0, 1[$ \\
            \hline
            \textbf{PWLCM} & $x_{n+1} = \begin{cases} \frac{x_n}{p} & 0 \leq x_n \leq p \\ \frac{x_n - p}{0.5 - p} & p \leq x_n \leq 0.5 \\ f(1 - x_n, p) & \text{ailleurs} \end{cases}$ & $p \in ]0, 0.5[, x_0 \in ]0, 1[$ \\
            \hline
            \textbf{Chebyshev} & $x_{n+1} = \cos(r_1 \arccos(x_n))$ & $r_1 \in \mathbb{N}$ \\
            \hline
            \textbf{Sinus} & $x_{n+1} = r_2 \sin(\pi x_n)$ & $r_2 \in ]0, 1]$ \\
            \hline
        \end{tabular}
    \end{table}
\end{frame}

\begin{frame}{Cartes Chaotiques - Bidimensionnelles (Slide 121)}
    \scriptsize
    \renewcommand{\arraystretch}{1.3}
    \begin{table}
        \centering
        \begin{tabular}{|l|c|l|}
            \hline
            \rowcolor{primary!20}
            \textbf{Carte} & \textbf{Expression} & \textbf{Paramètres} \\
            \hline
            \textbf{Logistique 2D} & 
            $\begin{cases} x_{n+1} = \mu_1 x_n(1-x_n) + \mu_2 y_n^2 \\ y_{n+1} = \mu_3 y_n(1-y_n) + \mu_4 x_n^2 \end{cases}$ & 
            $\mu_i \in \text{plages (ex: }[2.75, 3.4])$ \\
            \hline
            \textbf{Henon} & 
            $\begin{cases} x_{n+1} = 1 - a x_n^2 + y_n \\ y_{n+1} = b x_n \end{cases}$ & 
            $a=1.4, b=0.3, (x_0, y_0) \in ]0, 0.5]$ \\
            \hline
        \end{tabular}
    \end{table}
    
    \vspace{0.3cm}
    \begin{modernblock}{Complexité Accrue}
        Les cartes 2D offrent un espace de recherche plus complexe et une corrélation entrelacée entre les variables $x$ et $y$, renforçant la sécurité contre les attaques par retour sur l'état.
    \end{modernblock}
\end{frame}

\begin{frame}{Propriétés du Chaos Déterministe}
    \begin{columns}[T]
        \begin{column}{0.48\textwidth}
            \textbf{\textcolor{primary}{Propriétés Clés}}
            
            \begin{itemize}
                \item[\textcolor{success}{$\checkmark$}] \textbf{Sensibilité aux conditions initiales}
                \begin{itemize}
                    \footnotesize
                    \item Effet papillon
                    \item $\Delta x_0 = 10^{-14} \Rightarrow$ séquences totalement différentes
                \end{itemize}
                
                \vspace{0.2cm}
                \item[\textcolor{success}{$\checkmark$}] \textbf{Comportement pseudo-aléatoire}
                \begin{itemize}
                    \footnotesize
                    \item Déterministe mais imprévisible
                    \item Distribution uniforme
                \end{itemize}
                
                \vspace{0.2cm}
                \item[\textcolor{success}{$\checkmark$}] \textbf{Ergodicité}
                \begin{itemize}
                    \footnotesize
                    \item Exploration complète de l'espace
                    \item Pas de cycles courts
                \end{itemize}
            \end{itemize}
        \end{column}
        
        \begin{column}{0.48\textwidth}
            \textbf{\textcolor{primary}{Espace de Clés}}
            
            \small
            6 paramètres réels :
            \begin{itemize}
                \item $x_0^{(log)}, \mu$ (Logistique)
                \item $x_0^{(tent)}, r$ (Tente)
                \item $x_0^{(pwlcm)}, p$ (PWLCM)
            \end{itemize}
            
            \vspace{0.3cm}
            Précision : $10^{-14}$
            
            \vspace{0.2cm}
            \begin{alertblockbox}{Espace Total}
                \[
                (10^{14})^{6.4} \approx 2^{299} \text{ bits}
                \]
                
                \textbf{Comparaison :}
                \begin{itemize}
                    \footnotesize
                    \item AES-256 : $2^{256}$
                    \item Notre système : $\mathbf{2^{299}}$
                \end{itemize}
            \end{alertblockbox}
        \end{column}
    \end{columns}
\end{frame}

\begin{frame}{Schéma de l'Algorithme (Slide 134)}
    \centering
    \begin{tikzpicture}[scale=0.85, transform shape]
        % Inputs
        \node[draw, fill=secondary!50, minimum width=2cm] (img) at (0,3) {Image plain};
        \node[draw, fill=primary!10, rounded corners] (chaos) at (0,1) {Carte Logistique};
        
        % Process
        \node[draw, fill=accent!5] (vec) at (4,3) {Vectorisation (3NM)};
        \node[draw, fill=accent!5] (xor) at (7,3) {XOR ($X \oplus V$)};
        \node[draw, fill=success!5] (perm) at (10,3) {P-Box / S-Box};
        \node[draw, fill=success!5] (cbc) at (13,3) {Chaînage (CBC)};
        
        % Connections
        \draw[->, thick] (img) -- (vec);
        \draw[->, thick, primary] (chaos) -- (4,1) -- (4,2.5);
        \draw[->, thick] (vec) -- (xor);
        \draw[->, thick] (xor) -- (perm);
        \draw[->, thick] (perm) -- (cbc);
        
        % Final
        \node[draw, fill=danger!10, thick] (enc) at (13,1.5) {Image Chiffrée};
        \draw[->, thick] (cbc) -- (enc);
        
        % Legend
        \node[font=\tiny, right=0.1cm of chaos] {Vecteur chaotique V};
    \end{tikzpicture}
    
    \vspace{0.3cm}
    \begin{modernblock}{Flux de données}
        \centering \scriptsize
        Plain Image $\rightarrow$ Vectorisation (RGB) $\rightarrow$ Pré-diffusion $\rightarrow$ Confusion (S-Box) $\rightarrow$ Diffusion (CBC) $\rightarrow$ Cipher Image
    \end{modernblock}
\end{frame}

% --- Section Exemple Illustratif ---
\section{Exemple Illustratif Complet}

\begin{frame}{Architecture de l'Algorithme de Chiffrement}
    \centering
    \resizebox{0.9\textwidth}{!}{
    \begin{tikzpicture}[
        node distance=0.6cm and 1.2cm,
        auto,
        block/.style={rectangle, draw, fill=primary!10, text centered, align=center, rounded corners, minimum height=3em, minimum width=6em, font=\bfseries\footnotesize},
        vector/.style={circle, draw, fill=accent!10, text centered, align=center, font=\bfseries\tiny, minimum size=2.5em},
        arrow/.style={thick, ->, >=stealth, primary},
        chaos_bg/.style={draw, dashed, primary, fill=primary!5, rounded corners, inner sep=0.3cm}
    ]
        % Main Flow (Linear: Left to Right)
        \node [block] (input) {Image\\Originale};
        \node [block, right=of input] (prediff) {Pré-Diffusion\\(XOR)};
        \node [block, right=of prediff, xshift=0.5cm] (cbc) {Boucle CBC\\(Confusion)};
        \node [block, right=of cbc] (output) {Image\\Chiffrée};
        
        % Chaotic Parameters (Above)
        \node [vector, above=of prediff, yshift=0.4cm] (al) {AL};
        \node [vector, right=0.5cm of al, xshift=0.2cm] (sbox) {S-Box};
        \node [vector, above=of cbc, yshift=0.4cm] (c) {C};
        \node [vector, left=0.5cm of c] (bl) {BL};
        \node [vector, right=0.5cm of c] (cl) {CL};

        % Chaos Grouping Box (Moved to Background to avoid covering nodes)
        \begin{scope}[on background layer]
            \node [chaos_bg, fit=(al) (sbox) (bl) (c) (cl), label={[font=\bfseries\scriptsize\color{primary}]above:Générateurs Chaotiques}] (chaosbox) {};
        \end{scope}

        % Connections
        \draw [arrow] (input) -- (prediff);
        \draw [arrow] (prediff) -- (cbc);
        \draw [arrow] (cbc) -- (output);
        
        % Vertical Connections from Chaos
        \draw [arrow, dashed, thin] (al) -- (prediff);
        \draw [arrow, dashed, thin] (sbox) -- (cbc);
        \draw [arrow, dashed, thin] (bl) -- (cbc);
        \draw [arrow, dashed, thin] (c) -- (cbc);
        \draw [arrow, dashed, thin] (cl) -- (cbc);
    \end{tikzpicture}
    }
\end{frame}


\begin{frame}{Exemple - Paramètres Initiaux}
    \textbf{\textcolor{primary}{Configuration pour la démonstration}}
    
    \vspace{0.3cm}
    \begin{columns}[T]
        \begin{column}{0.48\textwidth}
            \textbf{Image simplifiée :}
            \begin{itemize}
                \item Taille : 2×2 pixels RGB
                \item Vecteur : 3×2×2 = 12 pixels
            \end{itemize}
            
            \vspace{0.3cm}
            \textbf{Clés chaotiques :}
            \begin{itemize}
                \footnotesize
                \item \textbf{Logistique :} $x_0 = 0.79878796$, $\mu = 3.756$
                \item \textbf{Tente :} $x_0 = 0.2$, $r = 1.99$
                \item \textbf{PWLCM :} $x_0 = 0.3$, $p = 0.254$
            \end{itemize}
        \end{column}
        
        \begin{column}{0.48\textwidth}
            \begin{modernblock}{Objectif}
                \begin{enumerate}
                    \footnotesize
                    \item Générer les séquences chaotiques
                    \item Calculer les vecteurs de contrôle
                    \item Construire la S-Box dynamique
                    \item Chiffrer pixel par pixel
                \end{enumerate}
            \end{modernblock}
        \end{column}
    \end{columns}
\end{frame}

\begin{frame}{Étape 1 - Génération des Séquences Chaotiques}
    \textbf{\textcolor{primary}{Développement des séquences u, v, w}}
    
    \vspace{0.3cm}
    \begin{columns}[T]
        \begin{column}{0.32\textwidth}
            \textbf{Carte Logistique (u)}
            \scriptsize
            \[
            u_0 = 0.79878796
            \]
            \[
            u_1 = 3.756 \times 0.799 \times (1-0.799)
            \]
            \[
            u_1 = 0.60352521
            \]
            \[
            u_2 = 0.89859
            \]
            \[
            \vdots
            \]
        \end{column}
        
        \begin{column}{0.32\textwidth}
            \textbf{Carte Tente (v)}
            \scriptsize
            \[
            v_0 = 0.2
            \]
            \[
            v_1 = 1.99 \times 0.2 = 0.398
            \]
            \[
            v_2 = 1.99 \times 0.398 = 0.79204
            \]
            \[
            \vdots
            \]
        \end{column}
        
        \begin{column}{0.32\textwidth}
            \textbf{Carte PWLCM (w)}
            \scriptsize
            \[
            w_0 = 0.3
            \]
            \[
            w_1 = \frac{0.3}{0.254} = 0.59055
            \]
            \[
            w_2 = 0.66929
            \]
            \[
            \vdots
            \]
        \end{column}
    \end{columns}
    
    \vspace{0.4cm}
    \begin{exampleblockbox}{Résultat}
        \scriptsize
        Génération de 12 valeurs pour chaque séquence (u, v, w)
    \end{exampleblockbox}
\end{frame}

\begin{frame}{Étape 2 - Vecteurs de Contrôle}
    \textbf{\textcolor{primary}{Conversion en vecteurs AL, BL, CL, C}}
    
    \vspace{0.3cm}
    \textbf{Formules :}
    \begin{align*}
    AL(i) &= \lfloor u(i) \times 10^{14} \rfloor \mod 256 \\
    BL(i) &= (\lfloor v(i) \times 10^{14} \rfloor \mod 256) \text{ OR } \mathbf{1} \quad \text{\textcolor{accent}{(Forcé impair)}} \\
    CL(i) &= \lfloor w(i) \times 10^{14} \rfloor \mod 256 \\
    C(i) &= \begin{cases} 1 & \text{si } u(i) \geq w(i) \\ 0 & \text{sinon} \end{cases}
    \end{align*}
    
    \vspace{0.3cm}
    \textbf{Exemple de calcul :}
    \begin{align*}
    u(0) &= 0.79878796 \\
    AL(0) &= \lfloor 0.79878796 \times 10^{14} \rfloor \mod 256 = 79878796000000 \mod 256 = \mathbf{224}
    \end{align*}
\end{frame}

\begin{frame}{Étape 2 - Développement des Vecteurs Pseudo-Aléatoires}
    \textbf{\textcolor{primary}{Génération du vecteur de contrôle $V(i)$ (Slide 126)}}
    
    \vspace{0.3cm}
    \begin{enumerate}
        \item \textbf{Séquences chaotiques} : $x(i) \in ]0, 1[$ générés par les cartes.
        \item \textbf{Transformation entière} :
        \[ V[i] = \text{int}(10^{14} \times x[i]) \pmod{256} \]
    \end{enumerate}
    
    \vspace{0.3cm}
    \begin{exampleblockbox}{Exemple (Slide 126)}
        \scriptsize
        $x_0 = 0.79878796 \Rightarrow V[0] = \lfloor 79878796... \rfloor \pmod{256} = \mathbf{224}$
    \end{exampleblockbox}
    
    \vspace{0.3cm}
    \small
    \renewcommand{\arraystretch}{1.2}
    \begin{table}
        \centering
        \begin{tabular}{|c|c|c|c|c|c|}
            \hline
            \rowcolor{primary!10}
            \textbf{i} & \textbf{u(i)} & $\dots$ & \textbf{AL(i) / V(i)} & \textbf{BL(i)} & \textbf{CL(i)} \\
            \hline
            0 & 0.7988 & $\dots$ & $\mathbf{224}$ & 33 & 76 \\
            1 & 0.6035 & $\dots$ & $\mathbf{152}$ & 101 & 149 \\
            \hline
        \end{tabular}
    \end{table}
\end{frame}

\begin{frame}{Étape 3 - Construction de la S-Box (Slide 131)}
    \textbf{\textcolor{primary}{Matrice de Substitution ST (256×256)}}
    
    \vspace{0.3cm}
    \begin{columns}[T]
        \begin{column}{0.55\textwidth}
            \scriptsize
            \renewcommand{\arraystretch}{1.2}
            \begin{table}
                \centering
                \begin{tabular}{|c|cccc|}
                    \hline
                    \rowcolor{primary!10}
                    \textbf{V(i) \textbackslash X(i)} & \textbf{0} & \textbf{1} & $\dots$ & \textbf{255} \\
                    \hline
                    \textbf{0} & $P_0$ & $P_1$ & $\dots$ & $P_{255}$ \\
                    \textbf{1} & $P_1$ & $P_2$ & $\dots$ & $P_0$ \\
                    \textbf{2} & $P_2$ & $P_3$ & $\dots$ & $P_1$ \\
                    \vdots & \vdots & \vdots & \ddots & \vdots \\
                    \textbf{255} & $P_{255}$ & $P_0$ & $\dots$ & $P_{254}$ \\
                    \hline
                \end{tabular}
                \caption{\tiny Look-up Table (ST)}
            \end{table}
        \end{column}
        
        \begin{column}{0.4\textwidth}
            \textbf{Génération Iterative :}
            \footnotesize
            \[ ST[0] = P = \text{argsort}(V) \]
            \[ ST[i, j] = ST[i-1, (j+i) \pmod{256}] \]
            
            \vspace{0.3cm}
            \textbf{Chiffrement (Confusion) :}
            \[ X'[i] = ST[V[i], X[i]] \]
        \end{column}
    \end{columns}
\end{frame}

\begin{frame}{Étape 4 - Vectorisation et Pré-diffusion}
    \textbf{\textcolor{primary}{Vectorisation de l'image (Slide 127-128)}}
    
    \vspace{0.3cm}
    \begin{columns}[T]
        \begin{column}{0.48\textwidth}
            \centering
            \begin{tikzpicture}[scale=0.6]
                \draw[thick] (0,0) rectangle (2,2) node[midway] {RGB};
                \draw[->, thick, accent] (2.5,1) -- (4,1);
                \draw[thick] (4,0.7) rectangle (8,1.3) node[midway] {$1 \times 3NM$};
            \end{tikzpicture}
            
            \vspace{0.3cm}
            \footnotesize
            \textbf{Méthode : Lecture en largeur}
            \begin{itemize}
                \item Extraction des canaux $R, G, B$
                \item Conversion en vecteur 1D
                \item Taille totale : $3 \times N \times M$
            \end{itemize}
        \end{column}
        
        \begin{column}{0.48\textwidth}
            \textbf{\textcolor{primary}{XOR avec vecteur chaotique}}
            
            \scriptsize
            \[ X(1, 3nm) \oplus V(1, 3nm) \Rightarrow X_{diff} \]
            
            \vspace{0.2cm}
            \begin{modernblock}{Formule}
                \centering
                $X_{diff}[i] = X[i] \oplus V[i]$
            \end{modernblock}
        \end{column}
    \end{columns}
\end{frame}
    \begin{frame}{Étape 4 - Vectorisation et Pré-diffusion (suite)}
    \begin{columns}[T]
        \begin{column}{0.48\textwidth}
            \textbf{Image originale (2×2 RGB) :}
            \scriptsize
            \[
            \begin{pmatrix}
            (120, 80, 200) & (50, 150, 30) \\
            (200, 100, 60) & (90, 180, 140)
            \end{pmatrix}
            \]
            
            \vspace{0.3cm}
            \textbf{Vecteur X (R, G, B) :}
            \scriptsize
            \[
            X = [120, 50, 200, 90, 80, 150, 100, 180, 200, 30, 60, 140]
            \]
        \end{column}
        
        \begin{column}{0.48\textwidth}
            \textbf{Pré-diffusion ($X \oplus AL$) :}
            \scriptsize
            \begin{align*}
            X[0] &= 120 \oplus 224 = \mathbf{152} \\
            X[1] &= 50 \oplus 152 = \mathbf{138} \\
            X[2] &= 200 \oplus 89 = \mathbf{145} \\
            &\vdots
            \end{align*}
            
            \vspace{0.3cm}
            \begin{modernblock}{Objectif}
                \footnotesize
                Diffuser l'information avant le chiffrement principal
            \end{modernblock}
        \end{column}
    \end{columns}
\end{frame}

\begin{frame}{Étape 5 - Chiffrement CBC}
    \textbf{\textcolor{primary}{Algorithme de diffusion séquentielle}}
    
    \vspace{0.3cm}
    \textbf{Initialisation (Vecteur IV) :}
    \[
    IV = (\sum AL + \sum BL + \sum CL) \mod 256
    \]
    
    \textbf{Algorithme de Chiffrement :}
    \begin{enumerate}
        \item \textbf{Premier pixel} ($i=0$) : $X[0] = X[0] \oplus IV$
        \item \textbf{Pixels suivants} ($i>0$) : $X[i] = X[i] \oplus X'[i-1]$
        \item \textbf{Transformation} : 
        \begin{itemize}
            \item Si $C[i] = 0 \Rightarrow X'[i] = S[AL[i]][X[i]]$
            \item Si $C[i] = 1 \Rightarrow X'[i] = (BL[i] \cdot X[i] + CL[i]) \pmod{256}$
        \end{itemize}
    \end{enumerate}
\end{frame}

\begin{frame}{Étape 5 (suite) - Calculs Détaillés}
    \textbf{\textcolor{primary}{Exemple pour les 3 premiers pixels}}
    
    \vspace{0.3cm}
    \scriptsize
    \textbf{Pixel 0 :}
    \begin{align*}
    X[0] &= 152 \oplus 142 = \mathbf{22} \\
    C[0] &= 1 \Rightarrow \text{Affine} \\
    X'[0] &= (33 \times 22 + 76) \mod 256 = (726 + 76) \mod 256 = \mathbf{34}
    \end{align*}
    
    \textbf{Pixel 1 :}
    \begin{align*}
    X[1] &= 138 \oplus 34 = \mathbf{168} \\
    C[1] &= 1 \Rightarrow \text{Affine} \\
    X'[1] &= (101 \times 168 + 149) \mod 256 = (16968 + 149) \mod 256 = \mathbf{221}
    \end{align*}
    
    \textbf{Pixel 2 :}
    \begin{align*}
    X[2] &= 145 \oplus 205 = 84 \\
    C[2] &= 1 \Rightarrow \text{Affine} \\
    X'[2] &= (201 \times 84 + 169) \mod 256 = \mathbf{78}
    \end{align*}
\end{frame}

\begin{frame}{Résultat Final de l'Exemple}
    \textbf{Vecteur chiffré obtenu}
    
    \vspace{0.3cm}
    \[
    X' = [\mathbf{34, 221,} \dots]
    \]
    
    \vspace{0.3cm}
    \textbf{Reconstruction en image 2×2 :}
    \scriptsize
    \[
    \begin{pmatrix}
    (166, 12, 88) & (205, 189, 156) \\
    (78, 45, 23) & (234, 201, 199)
    \end{pmatrix}
    \]
    
    \vspace{0.4cm}
    \begin{columns}[T]
        \begin{column}{0.48\textwidth}
            \begin{modernblock}{Original}
                \centering
                \footnotesize
                Image colorée structurée
            \end{modernblock}
        \end{column}
        
        \begin{column}{0.48\textwidth}
            \begin{alertblockbox}{Chiffrée}
                \centering
                \footnotesize
                Bruit aléatoire uniforme
            \end{alertblockbox}
        \end{column}
    \end{columns}
\end{frame}

% --- Section Implémentation ---
\section{Implémentation}

\begin{frame}{Architecture Full-Stack}
    \begin{columns}[T]
        \begin{column}{0.48\textwidth}
            \textbf{\textcolor{primary}{Stack Technologique}}
            
            \small
            \renewcommand{\arraystretch}{1.5}
            \begin{table}
                \centering
                \begin{tabular}{ll}
                    \toprule
                    \textbf{Couche} & \textbf{Technologie} \\
                    \midrule
                    Frontend & React + Vite \\
                    Backend & Flask (Python) \\
                    Crypto & NumPy + Pillow \\
                    Visualisation & Matplotlib \\
                    \bottomrule
                \end{tabular}
            \end{table}
        \end{column}
        
        \begin{column}{0.48\textwidth}
            \textbf{\textcolor{primary}{Fonctionnalités}}
            
            \begin{itemize}
                \footnotesize
                \item Upload d'images (PNG, JPEG, BMP, TIFF)
                \item Configuration des paramètres chaotiques
                \item Chiffrement/Déchiffrement en temps réel
                \item Génération de 30+ graphiques d'analyse
                \item Comparateur Avant/Après interactif
                \item Mode Clair/Sombre
            \end{itemize}
        \end{column}
    \end{columns}
\end{frame}

\begin{frame}{Déchiffrement - Algorithme Inverse}
    \begin{modernblock}{Logique de Déchiffrement}
        \small
        Le déchiffrement suit l'ordre inverse exact de l'encryption :
        \vspace{0.3cm}
        \begin{enumerate}
            \item \textbf{Régénération des outils} (Cartes Maps $\rightarrow$ Vecteurs $\rightarrow$ S-Box Inverse).
            \item \textbf{Inversion des Transformations} :
                \begin{itemize}
                    \footnotesize
                    \item Substitution Inverse via la S-Box inverse.
                    \item Affine Inverse : $X = (X' - CL) \times BL^{-1} \pmod{256}$.
                \end{itemize}
            \item \textbf{Annulation de la Diffusion} : On parcourt l'image de la fin vers le début ($X_i = Trans^{-1}(X'_i) \oplus X'_{i-1}$).
            \item \textbf{Annulation de la Pré-diffusion} : XOR final avec le vecteur $AL$.
        \end{enumerate}
    \end{modernblock}
    
    \vspace{0.4cm}
    \centering
    \textcolor{success}{\textbf{\checkmark Propriété d'un système symétrique bit-à-bit parfait}}
\end{frame}

% --- Section Analyse de Sécurité ---
\section{Analyse de Sécurité}

\begin{frame}{Tests NPCR \& UACI}
    \textbf{\textcolor{primary}{Mesures de diffusion et confusion}}
    
    \vspace{0.3cm}
    \begin{columns}[T]
        \begin{column}{0.48\textwidth}
            \textbf{Définitions :}
            
            \small
            \textbf{NPCR} (Number of Pixel Change Rate) :
            \[
            NPCR = \frac{1}{M \times N} \sum_{i,j} D(i,j) \times 100\%
            \]
            où $D(i,j) = \begin{cases} 1 & \text{si } C_1(i,j) \neq C_2(i,j) \\ 0 & \text{sinon} \end{cases}$
            
            \vspace{0.3cm}
            \textbf{UACI} (Unified Average Changing Intensity) :
            \[
            UACI = \frac{1}{M \times N} \sum_{i,j} \frac{|C_1(i,j) - C_2(i,j)|}{255} \times 100\%
            \]
        \end{column}
        
        \begin{column}{0.48\textwidth}
            \textbf{Résultats Obtenus :}
            
            \small
            \renewcommand{\arraystretch}{1.5}
            \begin{table}
                \centering
                \begin{tabular}{lcc}
                    \toprule
                    \textbf{Métrique} & \textbf{Résultat} & \textbf{Idéal} \\
                    \midrule
                    NPCR & \textcolor{success}{\textbf{100.00\%}} & $> 99.6\%$ \\
                    UACI & \textcolor{success}{\textbf{33.60\%}} & $\approx 33.46\%$ \\
                    \bottomrule
                \end{tabular}
            \end{table}
            
            \vspace{0.3cm}
            \begin{exampleblockbox}{Interprétation}
                \footnotesize
                Excellente sensibilité : un changement d'1 bit dans l'image originale modifie ~99.6\% des pixels chiffrés
            \end{exampleblockbox}
        \end{column}
    \end{columns}
\end{frame}

\begin{frame}{Entropie de Shannon}
    \textbf{\textcolor{primary}{Mesure du désordre}}
    
    \vspace{0.3cm}
    \textbf{Formule :}
    \[
    H(X) = -\sum_{i=0}^{255} p(x_i) \log_2 p(x_i)
    \]
    
    \vspace{0.3cm}
    \begin{columns}[T]
        \begin{column}{0.48\textwidth}
            \textbf{Résultats :}
            
            \small
            \renewcommand{\arraystretch}{1.5}
            \begin{table}
                \centering
                \begin{tabular}{lc}
                    \toprule
                    \textbf{Image} & \textbf{Entropie} \\
                    \midrule
                    Originale & ~7.2 bits/pixel \\
                    Chiffrée & \textcolor{success}{\textbf{7.9998}} bits/pixel \\
                    Idéal & 8.0 bits/pixel \\
                    \bottomrule
                \end{tabular}
            \end{table}
        \end{column}
        
        \begin{column}{0.48\textwidth}
            \begin{exampleblockbox}{Interprétation}
                \footnotesize
                L'entropie proche de 8.0 indique une distribution quasi-uniforme des pixels. \\
                \vspace{0.2cm}
                Cela signifie une excellente \textbf{confusion} : l'image chiffrée ressemble à du bruit aléatoire.
            \end{exampleblockbox}
        \end{column}
    \end{columns}
\end{frame}

\begin{frame}{Analyse de Corrélation}
    \textbf{\textcolor{primary}{Corrélation entre pixels adjacents}}
    
    \vspace{0.3cm}
    \textbf{Formule :}
    \[
    \rho_{xy} = \frac{\text{cov}(x, y)}{\sqrt{D(x)} \sqrt{D(y)}}
    \]
    
    \vspace{0.3cm}
    \textbf{Résultats :}
    
    \small
    \renewcommand{\arraystretch}{1.5}
    \begin{table}
        \centering
        \begin{tabular}{lccc}
            \toprule
            \textbf{Direction} & \textbf{Image Originale} & \textbf{Image Chiffrée} & \textbf{Réduction} \\
            \midrule
            Horizontale & 0.95 & \textcolor{success}{\textbf{0.0012}} & 99.87\% \\
            Verticale & 0.93 & \textcolor{success}{\textbf{0.0008}} & 99.91\% \\
            Diagonale & 0.91 & \textcolor{success}{\textbf{0.0015}} & 99.84\% \\
            \bottomrule
        \end{tabular}
    \end{table}
    
    \vspace{0.3cm}
    \begin{alertblockbox}{Conclusion}
        \footnotesize
        La corrélation spatiale est \textbf{totalement détruite}. \\
        Les pixels adjacents dans l'image chiffrée sont statistiquement indépendants ($\approx 0$).
    \end{alertblockbox}
\end{frame}

\begin{frame}{Analyse Visualisée - Corrélation}
    \centering
    \textbf{Distribution de corrélation (Image Chiffrée)}
    
    \vspace{0.2cm}
    \begin{columns}[T]
        \begin{column}{0.32\textwidth}
            \centering
            \footnotesize Horizontale (Red)
            \fbox{\includegraphics[width=\textwidth]{images/correlation/corr_enc_horizontal_red.png}}
        \end{column}
        \begin{column}{0.32\textwidth}
            \centering
            \footnotesize Verticale (Green)
            \fbox{\includegraphics[width=\textwidth]{images/correlation/corr_enc_vertical_green.png}}
        \end{column}
        \begin{column}{0.32\textwidth}
            \centering
            \footnotesize Diagonale (Blue)
            \fbox{\includegraphics[width=\textwidth]{images/correlation/corr_enc_diagonal_blue.png}}
        \end{column}
    \end{columns}
\end{frame}

% --- Section Résultats ---
\section{Résultats Expérimentaux}

\begin{frame}{Résultats Visuels - Exemple Lena}
    \centering
    \textbf{Image Lena (512×512)}
    
    \vspace{0.3cm}
    \begin{columns}[T]
        \begin{column}{0.32\textwidth}
            \centering
            \textbf{Originale}
            
            \vspace{0.2cm}
            \fbox{\includegraphics[width=\textwidth,height=0.5\textheight,keepaspectratio]{images/decrypted.png}}
            
            \vspace{0.2cm}
            \footnotesize
            Image claire et structurée
        \end{column}
        
        \begin{column}{0.32\textwidth}
            \centering
            \textbf{Chiffrée}
            
            \vspace{0.2cm}
            \fbox{\includegraphics[width=\textwidth,height=0.5\textheight,keepaspectratio]{images/encrypted.png}}
            
            \vspace{0.2cm}
            \footnotesize
            Bruit blanc uniforme
        \end{column}
        
        \begin{column}{0.32\textwidth}
            \centering
            \textbf{Déchiffrée}
            
            \vspace{0.2cm}
            \fbox{\includegraphics[width=\textwidth,height=0.5\textheight,keepaspectratio]{images/decrypted.png}}
            
            \vspace{0.2cm}
            \footnotesize
            Restauration parfaite
        \end{column}
    \end{columns}
    
    \vspace{0.3cm}
    \textbf{Temps d'exécution :} ~0.8 secondes
\end{frame}

\begin{frame}{Graphiques d'Analyse - Histogrammes RGB}
    \centering
    \textbf{Distribution des intensités de pixels (Original vs Chiffré)}
    
    \vspace{0.3cm}
    \begin{tikzpicture}[scale=0.75]
        % Ligne supérieure (Original)
        \node at (0,0) {\fbox{\includegraphics[width=4cm]{images/histograms/hist_orig_red.png}}};
        \node at (5.5,0) {\fbox{\includegraphics[width=4cm]{images/histograms/hist_orig_green.png}}};
        \node at (11,0) {\fbox{\includegraphics[width=4cm]{images/histograms/hist_orig_blue.png}}};
        
        % Ligne inférieure (Chiffré)
        \node at (0,-4.2) {\fbox{\includegraphics[width=4cm]{images/histograms/hist_enc_red.png}}};
        \node at (5.5,-4.2) {\fbox{\includegraphics[width=4cm]{images/histograms/hist_enc_green.png}}};
        \node at (11,-4.2) {\fbox{\includegraphics[width=4cm]{images/histograms/hist_enc_blue.png}}};
        
        % Labels
        \node[rotate=90, font=\small\bfseries, text=primary] at (-2.2,0) {ORIGINAL};
        \node[rotate=90, font=\small\bfseries, text=accent] at (-2.2,-4.2) {CHIFFRÉ};
    \end{tikzpicture}
    
    \vspace{0.4cm}
    \footnotesize
    La distribution devient parfaitement uniforme (plate) après chiffrement, détruisant toute information statistique.
\end{frame}
\begin{frame}{Analyse de Sécurité - Métriques}
    \vspace{0.1cm}
    
    \begin{columns}[T]
        \begin{column}{0.48\textwidth}
            \centering
            \textbf{\small Sensibilité aux Pixels}
            
            \vspace{0.2cm}
            \includegraphics[width=0.95\textwidth]{images/metrics/npcr_uaci_comparison.png}
            
            \vspace{0.1cm}
            \scriptsize
            \begin{tabular}{lcc}
                \hline
                \textbf{Métrique} & \textbf{Obtenu} & \textbf{Espéré} \\
                \hline
                NPCR & \textbf{100.00\%} & $> 99.60\%$ \\
                UACI & \textbf{33.60\%} & $\approx 33.34\%$ \\
                \hline
            \end{tabular}
            \vspace{0.2cm}
            \textcolor{success}{\checkmark Robustesse aux attaques différentielles}
        \end{column}
        
        \begin{column}{0.48\textwidth}
            \centering
            \textbf{\small Entropie de Shannon}
            
            \vspace{0.2cm}
            \includegraphics[width=0.95\textwidth]{images/metrics/entropy_comparison.png}
            
            \vspace{0.1cm}
            \scriptsize
            \begin{tabular}{lc}
                \hline
                \textbf{Image} & \textbf{Valeur Entropie} \\
                \hline
                Originale & $\approx$ 7.21 \\
                Chiffrée & \textbf{7.9998} \\
                \textbf{Idéale} & \textbf{8.0000} \\
                \hline
            \end{tabular}
            \vspace{0.2cm}
            \textcolor{success}{\checkmark Distribution uniforme (Bruit Blanc)}
        \end{column}
    \end{columns}
    
    \vspace{0.4cm}
    
    \begin{tcolorbox}[
        colback=primary!5!white,
        colframe=primary,
        boxrule=0.5pt,
        arc=3pt,
        left=5pt,
        right=5pt,
        top=3pt,
        bottom=3pt
    ]
        \scriptsize
        \textbf{Interprétation des résultats :}
        
        \begin{itemize}
            \item \textbf{NPCR (100\%)} : Un changement d'un seul pixel dans l'image originale modifie \textbf{tous} les pixels chiffrés
            \item \textbf{UACI (33.6\%)} : L'intensité moyenne des changements correspond à celle d'une attaque aléatoire idéale
            \item \textbf{Entropie (7.9998)} : L'image chiffrée possède une distribution quasi-parfaite, la rendant statistiquement indiscernable d'un bruit blanc aléatoire
        \end{itemize}
        
        \vspace{0.1cm}
        \centering
        \textcolor{success}{\textbf{\checkmark Toutes les métriques confirment la robustesse cryptographique du système}}
    \end{tcolorbox}
\end{frame}
\begin{frame}{Le Rôle des Cartes Chaotiques}
    \begin{columns}[T]
        \begin{column}{0.5\textwidth}
            \textbf{\textcolor{primary}{Pourquoi le Chaos ?}}
            \begin{itemize}
                \item \textbf{Génération de Clés} : Production de suites pseudo-aléatoires infatigables.
                \item \textbf{Sensibilité} : Une micro-variation de $x_0$ change tout le résultat.
                \item \textbf{Complexité} : Difficulté de prédiction sans les paramètres exacts.
            \end{itemize}
            
            \begin{exampleblockbox}{Utilisation dans l'Algorithme}
                \footnotesize
                Utilisées pour la permutation (P-Box), la substitution (S-Box) et les coefficients de diffusion affine.
            \end{exampleblockbox}
        \end{column}
        
        \begin{column}{0.45\textwidth}
            \centering
            \fbox{\includegraphics[width=\textwidth]{images/chaotic_maps/map_logistique.png}}
            \vspace{0.2cm}
            \fbox{\includegraphics[width=\textwidth]{images/chaotic_maps/map_pwlcm.png}}
            \footnotesize Diagrammes de bifurcation / Attracteurs
        \end{column}
    \end{columns}
\end{frame}

\begin{frame}{Synthèse des Résultats - Multi-Images (Slide 139-141)}
    \scriptsize
    \renewcommand{\arraystretch}{1.2}
    \begin{table}
        \centering
        \begin{tabular}{|l|c|c|c|c|}
            \hline
            \rowcolor{primary!10}
            \textbf{Image Test} & \textbf{Entropie ($\approx 8$)} & \textbf{NPCR ($> 99.6$)} & \textbf{UACI ($> 33.3$)} & \textbf{Corr. (H, V, D)} \\
            \hline
            \textbf{Female} & 7.9991 & 99.75\% & 33.44\% & -0.0008, 0.0007, -0.0002 \\
            \hline
            \textbf{House} & 7.9991 & 99.76\% & 33.47\% & 0.0002, -0.0026, 0.0009 \\
            \hline
            \textbf{Couple} & 7.9990 & 99.73\% & 33.52\% & 0.0014, 0.0015, -0.0061 \\
            \hline
            \textbf{Splash} & 7.9997 & 99.73\% & 33.48\% & -0.0005, 0.0008, -0.0001 \\
            \hline
        \end{tabular}
        \caption{Validation sur la base de données standard USC-SIPI}
    \end{table}
    
    \vspace{0.3cm}
    \begin{exampleblockbox}{Conclusion de l'Analyse}
        \footnotesize
        Les résultats sont \textbf{constants} sur différentes textures d'images et \textbf{conformes} aux espérances théoriques de la littérature cryptographique chaotique présentées dans le cours.
    \end{exampleblockbox}
\end{frame}

\begin{frame}{Comparaison avec l'État de l'Art}
    \small
    \renewcommand{\arraystretch}{1.5}
    \begin{table}
        \centering
        \begin{tabular}{lcccc}
            \toprule
            \textbf{Critère} & \textbf{AES} & \textbf{DES} & \textbf{Notre Système} & \textbf{Gain} \\
            \midrule
            Espace Clés & $2^{256}$ & $2^{56}$ & \textcolor{success}{\textbf{$2^{299}$}} & +43 bits \\
            NPCR & 99.60\% & 99.58\% & \textcolor{success}{\textbf{100.00\%}} & +0.4\% \\
            Entropie & 7.997 & 7.995 & \textcolor{success}{\textbf{7.9998}} & = \\
            Temps (512×512) & 0.5s & 0.3s & 0.8s & -0.3s \\
            S-Box & Statique & Statique & \textcolor{success}{\textbf{Dynamique}} & \checkmark \\
            \bottomrule
        \end{tabular}
    \end{table}
    
    \vspace{0.4cm}
    \begin{exampleblockbox}{Conclusion}
        \footnotesize
        Performance \textbf{compétitive} avec l'avantage majeur de la S-Box unique par image, rendant les attaques par dictionnaire impossibles.
    \end{exampleblockbox}
\end{frame}

% --- Conclusion ---
\section{Conclusion}

\begin{frame}{Synthèse des Contributions}
    \begin{columns}[T]
        \begin{column}{0.48\textwidth}
            \textbf{\textcolor{primary}{Réalisations}}
            
            \begin{itemize}
                \item[\textcolor{success}{\checkmark}] \textbf{Implémentation complète} de l'algorithme Projet 4
                \item[\textcolor{success}{\checkmark}] \textbf{Application Web Full-Stack} moderne et interactive
                \item[\textcolor{success}{\checkmark}] \textbf{Analyse exhaustive} (30+ métriques calculées)
                \item[\textcolor{success}{\checkmark}] \textbf{Visualisations} professionnelles exportables
                \item[\textcolor{success}{\checkmark}] \textbf{Support multi-formats} (PNG, JPEG, BMP, TIFF)
            \end{itemize}
        \end{column}
        
        \begin{column}{0.48\textwidth}
            \textbf{\textcolor{primary}{Résultats de Sécurité}}
            
            \small
            \renewcommand{\arraystretch}{1.5}
            \begin{table}
                \centering
                \begin{tabular}{lc}
                    \toprule
                    \textbf{Métrique} & \textbf{Valeur} \\
                    \midrule
                    Espace Clés & $2^{299}$ bits \\
                    NPCR & 100.00\% \\
                    UACI & 33.60\% \\
                    Entropie & 7.9998 bits/px \\
                    Corrélation & $< 0.002$ \\
                    \bottomrule
                \end{tabular}
            \end{table}
        \end{column}
    \end{columns}
\end{frame}

\begin{frame}{Limites et Perspectives}
    \begin{columns}[T]
        \begin{column}{0.48\textwidth}
            \textbf{\textcolor{danger}{Limitations Actuelles}}
            
            \begin{itemize}
                \footnotesize
                \item Temps de traitement pour images très grandes (>2048×2048)
                \item Transmission de l'IV non implémentée (IV dérivé des clés)
                \item Pas d'optimisation GPU
            \end{itemize}
        \end{column}
        
        \begin{column}{0.48\textwidth}
            \textbf{\textcolor{primary}{Améliorations Futures}}
            
            \begin{itemize}
                \footnotesize
                \item \textbf{Optimisation GPU} (CUDA) pour accélération
                \item \textbf{Chiffrement vidéo} (extension temporelle)
                \item \textbf{Application mobile} (React Native)
                \item \textbf{Compression avant chiffrement} (réduction taille)
                \item \textbf{Authentification} (HMAC, signatures)
            \end{itemize}
        \end{column}
    \end{columns}
\end{frame}

\begin{frame}{Conclusion Générale}
    \centering
    \vspace{0.5cm}
    
    \begin{exampleblockbox}{Message Principal}
        \centering
        \large
        Le chaos déterministe offre une alternative \textbf{puissante} et \textbf{élégante} aux méthodes cryptographiques traditionnelles pour la protection des images numériques.
    \end{exampleblockbox}
    
    \vspace{0.2cm}
    
    \begin{columns}[T]
        \begin{column}{0.48\textwidth}
            \begin{modernblock}{Points Forts}
                \footnotesize
                \begin{itemize}
                    \item Espace de clés gigantesque ($2^{299}$)
                    \item S-Box dynamique unique
                    \item Résultats de sécurité excellents
                    \item Interface moderne et intuitive
                \end{itemize}
            \end{modernblock}
        \end{column}
        
        \begin{column}{0.48\textwidth}
            \begin{alertblockbox}{Perspectives}
                \footnotesize
                \begin{itemize}
                    \item Extension au chiffrement vidéo
                    \item Optimisation pour IoT
                    \item Standardisation industrielle
                    \item Applications médicales
                \end{itemize}
            \end{alertblockbox}
        \end{column}
    \end{columns}
    
    \vspace{0.5cm}
    \textbf{\textcolor{primary}{Une solution data-driven pour la sécurité des images du futur}}
\end{frame}

\begin{frame}[plain]
    \begin{tikzpicture}[remember picture, overlay]
        \fill[primary] (current page.south west) rectangle (current page.north east);
        \node[text=white, align=center] at (current page.center) {
            {\Huge\bfseries Merci pour votre attention !}\\[1cm]
            {\Large Questions \& Discussions}\\[1.5cm]
            {\small Hicham Moussaid, Ahmed [NOM], Mohamed [NOM]}\\
            {\scriptsize Master SIE — ENS Meknès}
        };
    \end{tikzpicture}
\end{frame}

\end{document}
